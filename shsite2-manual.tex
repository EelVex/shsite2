\documentclass{memoir}

\title{shsite2 manual}
\author{eelvex}
\date{\today}

\makeglossary

\begin{document}
\begin{titlingpage}
\maketitle
\end{titlingpage}
\tableofcontents

\chapter{Commands} % {{{

	\section{\emph{make}} % {{{
	make [site]

	Invoke gnu-make command after some basic checks.

	A default Makefile is provided that is supposed to build a site from your sources.
	% }}}
	\section{\emph{publish}} % {{{
	publish [site]

	Publish your site through rsync,git,(s)ftp or other user-specified method.
	\section{\emph{info}}
	info <path | name | fullpathname | title | depends> <file>

	Get relevant info from file.
	% }}}
	\section{get-key}
	\section{blocks}
	\section{block-put}
	\section{parse}

% }}} commands

\chapter{parse-commands} % {{{
	%{{{
	Use the first line on commands that take multi-line input, for options.
	Separate options with spaces.

	\glossary{SHS2_PARSE_COMMANDS_PATH}{Path for default shs2 parse-commands}
	and \glossary{USER_PARSE_COMMANDS_PATH}{Path for user shs2 parse-commands} configuration variables point to paths where 
	parse-commands reside (default and user respectively).
	% }}}

	\section{table-tex}
	\section{code}

% }}} parse-commands

\chapter{Usage}
\chapter{Examples}

\printglossary

\end{document}
